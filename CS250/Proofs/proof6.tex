\documentclass[12pt]{article}
\usepackage{amsmath}
\usepackage{amssymb}
\usepackage{amsthm}
\usepackage{latexsym}
\usepackage{graphicx}
\usepackage{bm}
\usepackage{indentfirst}
\usepackage{booktabs} % for "\midrule" macro


\author{Molly Novash}
\title{%
\Huge Proof the Sixth for CS250  \\
\normalsize Powered by \LaTeXe}

\begin{document}
\sloppy
\maketitle
\bigskip

This document will provide proof of the following statement: \\


\textit{For any integer $n$, $n$ to the fourth power modified by 16 is equal to one.}
\bigskip

\begin{equation}
\forall n \in \mathbb{Z}((n = (2k+1), k \in \mathbb{Z}) \rightarrow ((n^4) \hspace{.1cm} \% \hspace{.1cm} 16 = 1))
\end{equation}
\bigskip
\begin{proof}

\begin{equation}
(n^4) \hspace{.1cm} \% \hspace{.1cm} 16 = 1
\end{equation}
\bigskip

Therefore, by algebra \ldots \\
\begin{equation}
(n^4 \hspace{.1cm} - 1) \hspace{.1cm} \% \hspace{.1cm} 16 = 0
\end{equation}
\bigskip

We need to prove that $(n^4 \hspace{.1cm} - 1)$ is a multiple of 16. \\

By factoring, we can produce the following equivalent expression \ldots \\
\begin{equation}
n^4 \hspace{.1cm} - 1 = (n^2 \hspace{.1cm} - 1)(n^2 \hspace{.1cm} + 1)
\end{equation}
\bigskip

In order to complete our proof efficiently, it is logical to choose substitutive expressions that will easily factor into 16. Since we will be squaring values of $n$, we would like to select a substitutive variable with a preceding 4, the square root of 16. \\

Example 4.4.6 on page 185 of the text "Discrete Mathematics with Applications, ed 6" by Susanna S. Epp, entitled "Representations of Integers Modulo 4," provides us with the following axiom: \\

Any integer can be written in one of the following four forms for some integer q \ldots

\begin{equation}
n = 4q \hspace{.2cm} or \hspace{.2cm} n = 4q + 1 \hspace{.2cm} or \hspace{.2cm} n = 4q + 2 \hspace{.2cm} or \hspace{.2cm} n = 4q + 3
\end{equation}
\bigskip

We know that $n$ is an odd integer. The first and third expressions, above, represent even integers, so we will use the second and fourth expressions as substitutes for $n$. \\

By expression (4), we can see that $(n^4 \hspace{.1cm} - 1)$ is equal to $(n^2 \hspace{.1cm} - 1)(n^2 \hspace{.1cm} + 1)$. We need to prove that $(n^4 \hspace{.1cm} - 1 )\hspace{.1cm} \% \hspace{.1cm} 16 = 0$. \\

If we can pull one factor of 16 evenly out of each $(n^2 \hspace{.1cm} - 1)$ and $(n^2 \hspace{.1cm} + 1)$ such that they multiply together to equal 16 times some integer expression, then we will have proven that $n^4 \hspace{.1cm} \%  \hspace{.1cm} 16 = 1$ for any odd integer $n$. \\

\textbf{\textit{Case 1:}} 4q + 1 \\

\textit{$n^2$ - 1:}
\begin{equation}
n^2  \hspace{.1cm} - 1 = [(4q + 1)(4q + 1)]  \hspace{.1cm} - 1 
\end{equation}
\begin{equation}
= 16q^2  \hspace{.1cm} + 8q \\
\end{equation}
\begin{equation}
\hspace{.1cm} = 8(2q^2  \hspace{.1cm} + q)
\end{equation}

\textit{$n^2$ + 1:}
\begin{equation}
n^2 \hspace{.1cm} + 1 = [(4q + 1)(4q + 1)] + 1
\end{equation}
\begin{equation}
= 16q^2 \hspace{.1cm} + 8q + 2
\end{equation}
\begin{equation}
\hspace{.2cm}= 2(8q^2 \hspace{.1cm} + 4q + 1)
\end{equation}
\bigskip

For case $4q \hspace{.1cm} + 1$ \ldots
\begin{equation}
(n^2 \hspace{.1cm} -1)(n^2 \hspace{.1cm} + 1) \hspace{.1cm} \% \hspace{.1cm} 16 = 0
\end{equation}
\begin{equation}
\hspace{.2cm}\equiv [8(2q^2 \hspace{.1cm} + q)][2(8q^2 \hspace{.1cm} + 4q + 1)] \hspace{.1cm} \% \hspace{.1cm} 16 = 0
\end{equation}
\begin{equation}
\hspace{.2cm}\equiv 16[(2q^2 \hspace{.1cm} + q)(8q^2 \hspace{.1cm} + 4q + 1)] \hspace{.3cm} \% \hspace{.1cm} 16 = 0
\end{equation}
\bigskip

Since the entire inner expression is a multiple of 2, let $r$ be an integer that equals $[(2q^2 \hspace{.1cm} + q)(8q^2 \hspace{.1cm} + 4q + 1)]$ by substitution. Thus \ldots \\

\begin{equation}
\equiv 16r \hspace{.1cm} \% \hspace{.1cm} 16 = 0, r \in \mathbb{Z}
\end{equation}
\bigskip

\textbf{\textit{Case 1:}} 4q + 3 \\

\textit{$n^2$ - 1:}
\begin{equation}
n^2  \hspace{.1cm} - 1 = [(4q + 3)(4q + 3)]  \hspace{.1cm} - 1 
\end{equation}
\begin{equation}
\hspace{.1cm}= 16q^2  \hspace{.1cm} + 24q + 8 \\
\end{equation}
\begin{equation}
\hspace{.2cm} = 8(2q^2  \hspace{.1cm} + 3q + 1)
\end{equation}

\textit{$n^2$ + 1:}
\begin{equation}
n^2 \hspace{.1cm} + 1 = [(4q + 3)(4q + 3)] + 1
\end{equation}
\begin{equation}
\hspace{.3cm}= 16q^2 \hspace{.1cm} + 24q + 10
\end{equation}
\begin{equation}
\hspace{.4cm}= 2(8q^2 \hspace{.1cm} + 12q + 5)
\end{equation}
\bigskip

For case $4q \hspace{.1cm} + 3$ \ldots
\begin{equation}
(n^2 \hspace{.1cm} -1)(n^2 \hspace{.1cm} + 1) \hspace{.1cm} \% \hspace{.1cm} 16 = 0
\end{equation}
\begin{equation}
\hspace{.2cm}\equiv [8(2q^2 \hspace{.1cm} + 3q + 1)][2(8q^2 \hspace{.1cm} + 12q + 5)] \hspace{.1cm} \% \hspace{.1cm} 16 = 0
\end{equation}
\begin{equation}
\equiv 16[(2q^2 \hspace{.1cm} + 3q + 1)(8q^2 \hspace{.1cm} + 12q + 5)] \hspace{.1cm} \% \hspace{.1cm} 16 = 0
\end{equation}
\bigskip

Since the entire inner expression is a multiple of 2, let $t$ be an integer that equals $[(2q^2 \hspace{.1cm} + 3q + 1)(8q^2 \hspace{.1cm} + 12q + 5)]$ by substitution. Thus \ldots \\

\begin{equation}
\equiv 16t \hspace{.1cm} \% \hspace{.1cm} 16 = 0, t \in \mathbb{Z}
\end{equation}
\bigskip

Thus, we have proven that for any odd integer $n$, $(n^4) \hspace{.1cm} \% \hspace{.1cm} 16 = 1$.
\end{proof}
\end{document}