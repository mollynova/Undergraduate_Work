\documentclass{article}
\usepackage{amsmath}
\usepackage{amssymb}
\usepackage{amsthm}
\usepackage{latexsym}
\usepackage{graphicx}
\usepackage{bm}
\usepackage{booktabs} % for "\midrule" macro


\author{Molly Novash}
\title{%
\Huge Proof the Second for CS250  \\
\normalsize Powered by \LaTeXe}

\begin{document}
\sloppy
\maketitle
\bigskip

Proof that $NAND$ is functionally complete in two parts  \ldots

\section{Proof: The Operator Set $\{$AND, OR, NOT$\}$ \\ is functionally complete.} 

Below is a table of the 16 logical operators, labelled numerically \ldots \\

Our aim is to progress through these operators in numerical order, and prove that 
each of them can be obtained by using nought but the operator set  $\{$AND, OR, NOT$\}$.

\bigskip
\begin{center}
\begin{tabular}{cccccccccccccccc}
1  & 2 & 3 & 4 & 5 & 6 & 7 & 8 & 9 & 10 & 11 & 12 & 13 & 14 & 15 & 16\\
\midrule
T & T & T & T & T & T & T & T & F & F & F & F & F & F & F & F\\
T & T & T & T & F & F & F & F & T & T & T & T & F & F & F & F\\
T & T & F & F & T & T & F & F & T & T & F & F & T & T & F & F\\
T & F & T & F & T & F & T & F & T & F & T & F & T & F & T & F\\
\end{tabular}
\end{center}

\bigskip
For each of the 16 logical operators, a truth table has been constructed for the purposes of this proof. In bold in the top-most row of each table is the logical expression from which each logical operator can be derived. The tables are as follows \ldots \\

\bigskip

\hspace*{-\parindent}%
\begin{minipage}{0.65\textwidth}
%\begin{enumerate}

\textbf{1.}\textit{ TTTT}\\
\begin{center}
\begin{tabular}{ccc}
$p$ & $q$ & $\bm{p\lor \lnot p}$\\
\midrule
T & T & T\\
T & F & T\\
F & T & T\\
F & F & $\smash{\underbrace{\text{T}}_{\textbf{(1)}}}$\\
\end{tabular}
\end{center}

\bigskip

\textbf{3.}\textit{ TTFT}\\
\begin{center}
\begin{tabular}{ccc}
$p$ & $q$ & $\bm{p\lor \lnot q}$\\
\midrule
T & T & T\\
T & F & T\\
F & T & F\\
F & F & $\smash{\underbrace{\text{T}}_{\textbf{(3)}}}$\\
\end{tabular}
\end{center}

\bigskip
\textbf{5.}\textit{ TFTT}\\
\begin{center}
\begin{tabular}{ccc}
$p$ & $q$ & $\bm{\lnot p \lor q}$\\
\midrule
T & T & T\\
T & F & F\\
F & T & T\\
F & F & $\smash{\underbrace{\text{T}}_{\textbf{(5)}}}$\\
\end{tabular}
\end{center}

\bigskip
\textbf{7.}\textit{ TFFT}\\
\begin{center}
\begin{tabular}{ccc}
$p$ & $q$ & $\bm{\lnot }$($\bm{ p \lor q}$)$\bm{\lor}$($\bm{p\land q}$)$$\\
\midrule
T & T & T\\
T & F & F\\
F & T & F\\
F & F & $\smash{\underbrace{\text{T}}_{\textbf{(7)}}}$\\
\end{tabular}
\end{center}

%\end{enumerate}
\end{minipage}%
\begin{minipage}{0.3\textwidth}
%\begin{enumerate}

\textbf{2.}\textit{ TTTF}\\
\begin{center}
\begin{tabular}{ccc}
$p$ & $q$ & $\bm{p\lor q}$\\
\midrule
T & T & T\\
T & F & T\\
F & T & T\\
F & F & $\smash{\underbrace{\text{F}}_{\textbf{(2)}}}$\\
\end{tabular}
\end{center}

\bigskip
\textbf{4.}\textit{ TTFF}\\
\begin{center}
\begin{tabular}{cc}
$\bm{p}$ & $q$\\
\midrule
T & T\\
T & F\\
F & T\\
$\smash{\underbrace{\text{F}}_{\textbf{(4)}}}$ & F\\
\end{tabular}
\end{center}

\bigskip
\textbf{6.}\textit{ TFTF}\\
\begin{center}
\begin{tabular}{cc}
$p$ & $\bm{q}$\\
\midrule
T & T\\
T & F\\
F & T\\
F & $\smash{\underbrace{\text{F}}_{\textbf{(6)}}}$\\
\end{tabular}
\end{center}

\bigskip
\textbf{8.}\textit{ TFFF}\\
\begin{center}
\begin{tabular}{ccc}
$p$ & $q$ & $\bm{p\land q}$\\
\midrule
T & T & T\\
T & F & F\\
F & T & F\\
F & F & $\smash{\underbrace{\text{F}}_{\textbf{(8)}}}$\\
\end{tabular}
\end{center}

%\end{enumerate}
\end{minipage}


% new page starts from here
\newpage
\hspace*{-\parindent}%
\begin{minipage}{.65\textwidth}

\bigskip
\textbf{9.}\textit{ FTTT}\\
\begin{center}
\begin{tabular}{ccc}
$p$ & $q$ & $\bm{\lnot }$($\bm{p \land q}$)$$\\
\midrule
T & T & F\\
T & F & T\\
F & T & T\\
F & F & $\smash{\underbrace{\text{T}}_{\textbf{(9)}}}$\\
\end{tabular}
\end{center}

\bigskip
\textbf{11.}\textit{ FTFT}\\
\begin{center}
\begin{tabular}{ccc}
$p$ & $q$ & $\bm{\lnot q}$\\
\midrule
T & T & F\\
T & F & T\\
F & T & F\\
F & F & $\smash{\underbrace{\text{T}}_{\textbf{(11)}}}$\\
\end{tabular}
\end{center}

\bigskip
\textbf{13.}\textit{ FFTT}\\
\begin{center}
\begin{tabular}{ccc}
$p$ & $q$ & $\bm{\lnot p}$\\
\midrule
T & T & F\\
T & F & F\\
F & T & T\\
F & F & $\smash{\underbrace{\text{T}}_{\textbf{(13)}}}$\\
\end{tabular}
\end{center}

\bigskip
\textbf{15.}\textit{ FFFT}\\
\begin{center}
\begin{tabular}{ccc}
$p$ & $q$ & $\bm{\lnot }$($\bm{p \lor q}$)$$\\
\midrule
T & T & F\\
T & F & F\\
F & T & F\\
F & F & $\smash{\underbrace{\text{T}}_{\textbf{(15)}}}$\\
\end{tabular}
\end{center}

\end{minipage}
\begin{minipage}{.3\textwidth}

\bigskip
\textbf{10.}\textit{ FTTF}\\
\begin{center}
\begin{tabular}{ccc}
$p$ & $q$ & $\bm{\lnot }$($\bm{ p \land q}$)$\bm{\land}$($\bm{p\lor q}$)$$\\
\midrule
T & T & F\\
T & F & T\\
F & T & T\\
F & F & $\smash{\underbrace{\text{F}}_{\textbf{(10)}}}$\\
\end{tabular}
\end{center}

\bigskip
\textbf{12.}\textit{ FTFF}\\
\begin{center}
\begin{tabular}{ccc}
$p$ & $q$ & $\bm{p \land \lnot q}$\\
\midrule
T & T & F\\
T & F & T\\
F & T & F\\
F & F & $\smash{\underbrace{\text{F}}_{\textbf{(12)}}}$\\
\end{tabular}
\end{center}

\bigskip
\textbf{14.}\textit{ FFTF}\\
\begin{center}
\begin{tabular}{ccc}
$p$ & $q$ & $\bm{\lnot p \land q}$\\
\midrule
T & T & F\\
T & F & F\\
F & T & T\\
F & F & $\smash{\underbrace{\text{F}}_{\textbf{(14)}}}$\\
\end{tabular}
\end{center}

\bigskip
\textbf{16.}\textit{ FFFF}\\
\begin{center}
\begin{tabular}{ccc}
$p$ & $q$ & $\bm{p \land \lnot p}$\\
\midrule
T & T & F\\
T & F & F\\
F & T & F\\
F & F & $\smash{\underbrace{\text{F}}_{\textbf{(16)}}}$\\
\end{tabular}
\end{center}

\end{minipage}

\bigskip
\bigskip
\bigskip
\hspace*{-\parindent}%
Conclusion: The Operator Set $\{$AND, OR, NOT$\}$ is functionally complete.  $\square$

\newpage
\section{Proof: $NAND$ is functionally complete.}

The previous section provides proof that the Operator Set $\{$AND, OR, NOT$\}$ is functionally complete. Therefore, if it is possible to duplicate the effects of all three elements in that Operator Set using only the operator $NAND$, $NAND$ is also functionally complete. 

\subsection{Proof: The $NAND$-only equivalent of $AND$}
\begin{proof}
\bigskip

Below is the truth table for the $AND$ operator, juxtaposed with a truth table which generates the same binary values using only the $NAND$ operator.

\bigskip
\subsubsection{NAND-only version of P AND Q}
\bigskip
\begin{center}
\begin{tabular}{cccc}
$p$ & $q$ & $p \hspace{.1cm}| \hspace{.1cm}q$ & $(p \hspace{.1cm}|\hspace{.1cm}q) \hspace{.1cm} | \hspace{.1cm}(p \hspace{.1cm} | \hspace{.1cm} q)$\\
\midrule
T & T & F & T\\
T & F & T & F\\
F & T & T & F\\
F & F & T & $\smash{\underbrace{\text{F}}_{\textbf{($ \hspace{.1cm} |\hspace{.1cm} of \land$)}}}$\\
\end{tabular}
\end{center}

\bigskip
\subsubsection{Q AND P}
\bigskip
\begin{center}
\begin{tabular}{ccc}
$p$ & $q$ & $\bm{q \land p}$\\
\midrule
T & T & T\\
T & F & F\\
F & T & F\\
F & F & $\smash{\underbrace{\text{F}}_{\textbf{($\land$)}}}$\\
\end{tabular}
\end{center}
\bigskip
\bigskip
Conclusion: $AND$ can be constructed from $NAND$.
\end{proof}

\subsection{Proof: The $NAND$-only equivalent of $OR$}
\begin{proof}
\bigskip

Below is the truth table for the $OR$ operator, juxtaposed with a truth table which generates the same binary values using only the $NAND$ operator.

\bigskip
\subsubsection{NAND-only version of P OR Q}
\bigskip
\begin{center}
\begin{tabular}{cccccc}
$p$ & $q$ & $p \hspace{.1cm}|\hspace{.1cm} q$ & $q \hspace{.1cm}|\hspace{.1cm} (p \hspace{.1cm} | \hspace{.1cm} q) $ & $p \hspace{.1cm} | \hspace{.1cm}(q \hspace{.1cm}|\hspace{.1cm} (p \hspace{.1cm} | \hspace{.1cm} q))$ & $(q \hspace{.1cm}|\hspace{.1cm} (p \hspace{.1cm} | \hspace{.1cm} q)) \hspace{.1cm} | \hspace{.1cm}(p \hspace{.1cm} | \hspace{.1cm} (q \hspace{.1cm}|\hspace{.1cm} (p \hspace{.1cm} | \hspace{.1cm} q)))$\\
\midrule
T & T & F & T & F & T\\
T & F & T & T & F & T\\
F & T & T & F & T & T\\
F & F & T & T & T & $\smash{\underbrace{\text{F}}_{\textbf{($ \hspace{.1cm} |\hspace{.1cm} of \lor$)}}}$\\
\end{tabular}
\end{center}

\bigskip
\subsubsection{OR P}
\bigskip
\begin{center}
\begin{tabular}{ccc}
$p$ & $q$ & $\bm{\lnot p}$\\
\midrule
T & T & T\\
T & F & T\\
F & T & T\\
F & F & $\smash{\underbrace{\text{F}}_{\textbf{($\lor$)}}}$\\
\end{tabular}
\end{center}
\bigskip
\bigskip
Conclusion: $OR$ can be constructed from $NAND$.
\end{proof}


\subsection{Proof: The $NAND$-only equivalent of $NOT$}
\begin{proof}
\bigskip

Below is the truth table for the $NOT$ operator, juxtaposed with a truth table which generates the same binary values using only the $NAND$ operator.

\bigskip
\subsubsection{NAND-only version of NOT P}
\bigskip
\begin{center}
\begin{tabular}{ccccc}
$p$ & $q$ & $p \hspace{.1cm}|\hspace{.1cm} q$ & $q \hspace{.1cm}|\hspace{.1cm} (p \hspace{.1cm} | \hspace{.1cm} q) $ & $p \hspace{.1cm} | \hspace{.1cm} ( q \hspace{.1cm} | \hspace{.1cm} (p \hspace{.1cm} | \hspace{.1cm} q))$\\
\midrule
T & T & F & T & F\\
T & F & T & T & F\\
F & T & T & F & T\\
F & F & T & T & $\smash{\underbrace{\text{T}}_{\textbf{($ \hspace{.1cm} |\hspace{.1cm} of \lnot$)}}}$\\
\end{tabular}
\end{center}

\bigskip
\subsubsection{NOT P}
\bigskip
\begin{center}
\begin{tabular}{ccc}
$p$ & $q$ & $\bm{\lnot p}$\\
\midrule
T & T & F\\
T & F & F\\
F & T & T\\
F & F & $\smash{\underbrace{\text{T}}_{\textbf{($\lnot$)}}}$\\
\end{tabular}
\end{center}
\bigskip
\bigskip
Conclusion: $NOT$ can be constructed from $NAND$.
\end{proof}

\bigskip
\hspace*{-\parindent}%
Final Conclusion: $NAND$ is functionally complete. $\square$

\end{document}