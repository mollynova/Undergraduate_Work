\documentclass{article}
\usepackage{amsmath}
\usepackage{amssymb}
\usepackage{amsthm}
\usepackage{latexsym}
\usepackage{graphicx}

\author{Molly Novash}
\title{%
\Huge Proof the First for CS250  \\
\normalsize Powered by \LaTeXe}
\begin{document}
\maketitle
Consider the relation f over the real numbers \; ,
\begin{equation}
f: \mathbf{R} \longrightarrow \mathbf{R} \; , where \; f(x) \; = \; x^3 \; \ldots
\end{equation}

\section{Prove or disprove: $f$ is a function.}
\begin{proof} 

A relation 
\emph{F} from 
\emph{A} to 
\emph{B} is a function, if and only if:\\

1. Every element of $A$ is the first element of an ordered pair of $F$.

2. No distinct ordered pairs in $F$ have the same first element.\\

Any real number, if raised to an odd power, retains its positivity or negativity. Therefore \; \ldots

\begin{equation}
\forall (x,y) \in F \; | \; \{isUnique(x)\}
\end{equation} \\


\ldots satisfying property (1).\\

Since the the set of the first element in each ordered pair of F is equal to the real numbers, $\mathbf{R}$, which is continuuous from -
$\infty$
to 
$\infty$, there can be no ordered pairs in $F$ that have the same first element.\\

\ldots satisfying property (2).\\

Conclusion: $f$ is a function.
\end{proof}

\section{Prove or disprove: $f$ is injective (one-to-one).}
\begin{proof} 

A function is injective (one-to-one) if every element of the codomain is mapped to by at most one element of the domain. 

One method of testing whether or not a function is injective is the performance of a horizontal line test on a graph of the function in question. If you can draw a horizontal line at any place on the graph which touches the function in two places or more, the function is not injective. \\

Below is a graph of the function, $f(x) = x^3$ \\

\hfill\includegraphics{graph1.jpg}\hspace*{\fill} \\

As indicated by the graph above, there is no horizontal line that can possibly be drawn on the graph of $f(x) = x^3$ which would touch the function in multiple places.

Given the erroneous nature of images, it would be possible to argue from the graph alone that a horizontal line drawn at $y = 0$ would pass through the function more than once. \\

Consider the function $f(x) = x^3$. \\
\begin{equation}
When \; x = 0 \; , \; f(0) = 0^3 = 0
\end{equation} \\

There is no other possible value for $x$ such that $x^3 = 0$.\\

Conclusion: $f$ is injective.

\end{proof}

\section{Prove or disprove: $f$ is surjective (onto).}
\begin{proof}

A function $f$ (from set $A$ to $B$) is surjective if and only if for every $y$ in $B$, there is at least one $x$ in $A$ $\|$ $f(x) = y$. In other words, $f$ is surjective if and only if $f(A) = B$. 

\begin{equation}
f(x) = x^3 = y , \qquad x , y \in \mathbf{R} \ldots \\
\end{equation}

\begin{equation}
\therefore \;
\sqrt[3]{x^3} = \sqrt[3]{y}\qquad \\
\end{equation}

\begin{equation}
\therefore \; 
x = y^{1/3} \\
\end{equation}

\begin{equation}
\therefore \; \forall x \in \mathbf{R} \; \exists y \; \| f(x) = y  \\
\end{equation}

Conclusion: $f$ is surjective.

\end{proof}

\section{Prove or disprove: $f$ is bijective (both one-to-one and onto).}
\begin{proof}

A function is considered bijective if it has been proven to be both injective and surjective. We have previously concluded in this document that $f(x) = x^3$ is a function, and that it is both injective and surjective. \\

Conclusion: $f$ is bijective.
\end{proof}

\end{document}